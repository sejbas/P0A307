    \subsection{Hv-spørgsmål}
    
    Da flere og flere enheder bliver introduceret til et individs lokale netværk i form af IoT-enheder, vil dette også inkludere flere indgangspunkter eller ”huller”, hvor en personlig invadering kunne finde sted. Specielt produkter der er designet til enkelt praktiske formål, er potentielle svaghedspunkter i et lokalt netværk, da deres sikkerheds standarter generelt er underlegen produkter med mere generelle formål. IoT enheder bliver i en stigende grad mere og mere populære. Markedet for IoT enheder har vokset voldsomt inden for de seneste år, og denne stigning i efterspørgslen ser kun ud til at vokse ydereligere i fremtiden. Der er allerede ifl. Statista 23,14 milliarder enheder i brug, og antallet vil cirka stige 3 gange inden år 2025. Derfor er det et meget relevant problem, da det efterhånden vil være en del af den normale borgers hverdag. Netop pga. det stigende marked, er der en meget udbredt konkurrence omkring disse produkter. Der er en stor interesse i at være de første på markedet med de nyeste produkter, da teknologien er konstant forbedrende. Dette kan gøre at sikkerheden af produkterne ofte ikke sættes i fokus, da man i stedet vil udgive produkterne hurtigst og billigst muligt. Desværre åbner dette fænomen flere og flere mennesker op for cyber-angreb. Eksperter i IT verden over advare imod de potentielle farer der opstår ved IoT enheder.\autocite{Rainie2017}\\
    
    En invasion af det lokale netværk, kan have alvorlige konsekvenser i form af tyveri af personlig data, krænkelse af personlig data, overvågning, fysisk skade (da mange af enhederne kan kontrolleres trådløst igennem internettet) osv.\\
    
    Der er flere grunde til det daværende og voksende sikkerheds problem. Det er eks. en markedsfejl at sikkerheds nedprioriteres fremfor at udgive produktet hurtigst og billigst muligt. Der er et manglende incitament til at opgradere sikkerhedsstandarterne i produkterne, da først til mølle princippet, spiller så afgørende en rolle i produktets succes. Desuden kan man sætte spørgsmålstegn til de lovpligtige sikkerhedsstandarterne af produkterne. Specielt kinesiske produkter har vist sig at være sårbare overfor cyber-angreb, fordi deres sikkerhed har vist sig at komprimere meget let, bl.a. fordi at de kinesiske varer sætter et stort fokus på discount frem for kvalitet. Der er ingen lovgivning for importerede sikkerhedsstandarter i Europa, hvilket betyder at de importerede enheder kun er underlagt de sikkerhedskrav, der finder sted i deres oprindelige lands lovgivning.\\