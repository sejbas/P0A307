\makeglossaries

\newglossaryentry{DDoS_angreb}
{
    name={DDoS angreb},
    description={(Distributed Denial of Service angreb) er et angreb der gør brug af adskillige enheder til at sende en stor mængde pakker til en internetserver for at overbelaste den. Formålet med DDoS angreb er at gøre en internetserver utilgængelig for brugere.  Side: },
    plural={DDoS angreb}
}

\newglossaryentry{zero-day_exploit}
{
    name={zero-day exploit},
    description={Er et exploit som ikke er kendt af dem der ville være interesseret i at fikse den. Da disse exploits benytter sig af sårbarheder som endnu ikke er kendte, er de ekstremt farlige, og det er oftest denne type exploit der bruges mod ellers sikre systemer},
    plural={zero-day exploits}
}

\newglossaryentry{SQL_injection}
{
    name={SQL injection},
    description={Er et angreb der målretter en SQL server, oftest bag et website. Dette sker f.eks ved user input i et felt der ikke er ordentligt sikret, og derfor lader en bruger indtaste SQL kommandoer som så vil blive eksekveret på serveren. Sider med gode sikkerhedsstandarder vil filtrere brugerinput mod kommandoer som dette.  Side: },
    plural={SQL injections}
}

\newglossaryentry{cross_site_scripting}
{
    name={cross-site scripting},
    description={Er et angreb hvor et ondsindet script bliver injected på en ellers normal hjemmeside. Dette lader en angriber sende og eksekvere det ondsindede script på en anden brugers computer. Side: },
    plural={cross-site scripting}
}

\newglossaryentry{botnet}
{
    name={botnet},
    description={Er et antal enheder forbundet til internettet der indeholder bots. Et botnet vil ofte bestå af computere der tilhører andre end ejeren af botnettet, men som er blevet overtaget af malware. Ejeren af botnettet kan så benytte alle disse enheder til f.eks \Gls{DDoS_angreb}, hvor alle enhederne vil blive sat til at sende pakker til den samme internetserver. Side: }, 
    plural={botnets}
}

\newglossaryentry{vulnerability-database}
{
    name={vulnerability-database},
    description={Er en database sat op med kendte software, kan der være velkendte sikkerhedshuller i det system den køre. Mange af disse er offtentligt tilgængelige og søgbare på internettet.  Side: }, 
    plural={vulnerability-databaser}
}

\newglossaryentry{salinitet}
{
    name={salinitet},
    description = {Saltindholdet i vand. Side: }
}

\newglossaryentry{DNS_rebinding}
{
    name={DNS rebinding},
    description={Er en type angreb hvor angriberen for en bruger til at køre et ondsindet client-side script gennem et website, som angriber andre enheder på offerets netværk.  Side: }, 
    plural={DNS rebindings}
}

\newglossaryentry{GET}
{
    name={GET},
    description={Er en type HTTP request, der bruges til at anmode om data fra et website.  Side: }, 
    plural={GET}
}

\newglossaryentry{POST}
{
    name={POST},
    description={Er en type HTTP request, der bruges til at sende data til et website.  Side: }, 
    plural={POST}
}