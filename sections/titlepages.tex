\selectlanguage{danish}
\pdfbookmark[0]{Danish title page}{label:titlepage_da}
\aautitlepage{%
  \danishprojectinfo{
    P0 projekt - Internet of Things %title
  }{%
    P0 Hvis programmer er løsningen - hvad er så problemet?%theme
  }{%
    Efterårssemestret 2018 %project period
  }{%
    A307a % project group
  }{%
    %list of group members
    Alexander Nykjær\\ 
    Mostaan Hashemi\\
    Mark Ambjørn Christiansen\\
    Floris Leferink\\
    Julius Nyborg Olesen\\
    Sten Kirk Larsen\\
    Christoffer Aaen
  }{%
    %list of supervisors
    Mads Vestergård Carlsen
  }{%
    1 % number of printed copies
  }{%
    \today % date of completion
  }%
}
{%department and address
  \textbf{Datalogi og Software}\\
  Aalborg Universitet\\
  \href{http://tnb.aau.dk}{http://tnb.aau.dk}
}{% the abstract
    In this report the group was tasked to find, and analyse a problem with the headline: "If coding is the solution, what is the problem" and with the topic: "Internet of Things"\\
    The report will cover the considerations done by the group with the purpose of exposing the flaws of the IoT devices we see on the market. The report focususes on how the weak security on IoT devices would affect the rest of the devices on the LAN. With this in mind the repport considers both real life examples of badly secured IoT-devices, that had devastating consequences for bigger companies, as well as smaller ones and a scientific paper that explained a specific, but relatively simple way of getting access to IoT devices if the devices are not secured properly.
  

  %to find the problem which the group would continue working on if the project had continued.
}