
\chapter{Rapport}

    \section{Valg af emne}
    
    Vi har valgt at arbejde med Internet of Things, da dette emne bliver stadig mere relevant i moderne samfund. Hjemmene fyldes af flere og flere enheder der alle bliver koblet på internettet på den ene eller anden måde \autocite{website:Statistica2015}.\\
    Over de sidste ti år er der blevet tilføjet en lang række enheder til danskernes netværk. Det er ikke længere kun computere og printere der er tilkoblet nettet, nu rækker det sig også til telefoner, smartwatches, Chromecasts, fjernsyn og så videre. \\ \todo{Ikke kun danskere, evt ændre sætning. Derudover er disse ting måske ikke de mest relavante at nævne.}
    Listen over ting der bliver tilkoblet netværk omkring i verden er stadig stigende og denne trend ser ud til kun at fortsætte. Eftersom den vestlige verden bliver mere og mere afhængig af elektronik og dets "Always online"\ fordele, vil der også blive produceret flere og flere af sådanne produkter på markedet, for at tjene på den stadig stigende efterspørgsel efter "smarte"\ produkter. Listen af IoT produkter strækker sig efterhånden fra køleskabe til peberkværn, og så videre til toiletter. Alt bliver koblet op til internettet og kan styres ved et tryk på en app. Alt dette lyder som en utopi, en verden hvori alt kan styres ved et tryk på en telefon, hvor man ikke længere skal rejse sig for at sætte kaffen over eller slukke for komfuret.\\
    Problemet ligger dog lige netop i alle de enheder vi tilkobler vores netværk. Mange af disse enheder er lavet for at tjene penge på det stigende marked, for at komme ud på markedet før konkurrenterne, dette betyder ofte også at sikkerheden på disse enheder bliver nedprioriteret for at få et produkt ud på markedet hurtigere end konkurrenterne.\\
    Herved bliver der tilføjet en stor mængde af usikre enheder til ens netværk, hvilket giver et større potentiale for at ens lokale netværk bliver sårbart overfor uvedkommende enheder. Ovenpå at der bliver forbundet stadig flere enheder til ens netværk, er der i følge Forbes en stigning på 280\% i den første halvdel af 2017 på angreb rettet mod IoT enheder \autocite{Forbes2018}.\\
    Dette vil altså sige at disse enheder der bliver tilkoblet folks netværk, udgør en stadigt stigende trussel mod individernes egne netværk. Denne stigning af enheder gør sig ikke kun gældende i privaten, men bygger også ud imod en bredere udnyttelse af disse enheder i erhvervssektore af forskellige slags.
    
        \subsection{Beskrivelse af IoT}
        
        Begrebet Internet of Things (IoT) blev skabt i 1999 af MIT Auto-ID Center grundlæggere, Kevin Ashton og David L. Brock.\autocite{Hashmi2017} Auto-ID er en bred vifte af teknologier brugt i industrier til at øge effektiviteten, automation samt reducere fejl. Disse teknologier kan være sensore, stemme genkendelse, biometri etc. \todo{ECT er englsk, og biometri er taget direkte fra engelsk? Etc. ordnet https://ordnet.dk/ddo/ordbog?query=etc. Biometri i ordnet https://ordnet.dk/ddo/ordbog?query=biometri}
        Siden 2003 har Auto-ID teknologi ændret sig til hovedsageligt at være Radio Frequency Identification(RFID).\autocite{Sundmaeker2010}. Oprendeligt var MIT Auto-ID Center industri sponsoreret forsknings og udviklingscenter, hvor alt deres arbejde blev frit tilgængeligt. Centeres vision var:\\
        \textit{``The Auto-ID Center envisions a world in which all electronic devices are networked and every object, whether it is physical or electronic, is electronically tagged with information pertinent to that object.  We envision the use of physical tags that allow remote, contactless interrogation of their contents; thus, enabling all physical objects to act as nodes in a networked physical world. The realization of our vision will yield a wide range of benefits in diverse areas including supply chain management and inventory control, product tracking and location identification, and human-computer and human- object interfaces. Our vision of ubiquitous automated identification technologies and their applications drives our research agenda and goals.''}- \autocite[Kapitel 2,p. ~4]{Sarma2001} \\
        I oktober 2003 skiftede MIT Auto-ID Center navn til Cambridge Auto-ID Lab og blev lukket. Centeret blev delt i til Auto-ID Labs forsknings enhed og EPCglobal kommercielt enhed eget af UCC og EAN.\autocite{Sundmaeker2010}
        \\Formålet med Auto-ID Labs i dag er at udvikle et netværk, som forbinder computer til objekter. Dette er ikke blot hardware (RFID tag og læsere)\todo{Evt forklaring af dette?}
        eller software som køre på et netværk, men alt hvad der benyttes til at skabe IoT. Det vil sige hardware, netværk software og protokoller, sprog til at beskrive objekter således at computer kan kommunikere. Det er ikke et nyt internet, men elementer bygget oven på eksisterende internet teknologi, som gøre det muligt at spore og dele information på tværs af internettet.\autocite{Sundmaeker2010} \\
        I denne rapport mener vi med netværk, et lokalt netværk omkring en eller flere roterer tilkoblet samme modem. \todo{Kan vi skrive det her sammen med ovenstående eller skal det flyttes andet sted?}
        \\
  
    
            \subsubsection{Indskrænkning af IoT}
            %%Omskrevet sektion, skal denne gemmes i stedet for den nedenstående?
            Internet of Things er i den moderne verden blevet et udtryk der dækker over alle enheder der er tilkoblet internettet på den ene eller anden måde.\\
            Da denne mængde af enheder bliver en for stor målgruppe for gruppen at arbejde med i dette projekt har gruppen valgt at lave en definition af IoT som gør sig gældende i resten af denne rapport. \\
            Til at starte med ekskluderedes alle "General purpose devices"\ altså: Telefoner, printere, computere og lignende enheder. \\
            General Purpose Devices(GPD) er ofte produceret af større firmaer som Hp, Samsung, Apple el.lign, disse er firmaer der har ressourcerne til at lave regulære opdateringer til deres enheder på en rimelig regulær basis. Blandt disse opdateringer følger der også løbende sikkerheds opdateringer, der lapper store dele af de huller der er blevet fundet i deres software.\\
            Dette betyder at definitionen i rapporten af IoT enheder omfatter de enheder der er nye på markedet, og ofte ikke produceret med sikkerhed som højeste prioritet. IoT enheder i rapporten er altså begrænset til ting der kun nyligt er blevet tilkoblet internettet, så som: Kaffemaskiner, toiletter og køleskabe. \\
            
            
            
            %%Gamle sektion
            \todo{Gammel sektion}
            Da definitionen på Internet of Things er ret bred, har vi valgt at indskrænke vores definition af IoT enheder. Vi har valgt til først at ekskludere General purpose devices (Dette inkluderer ting som: Pc' er, printere og mobil telefoner) da disse enheders sikkerhed er i stort fokus hos producenterne, og disse enheder får regulære opdateringer. Dette betyder ikke at enhver enhed der er bygget med sikkerhed
            in mente ikke kan være en IoT enhed, blot at vores fokus er på de mange nye usikre enheder.\\
            Vi definere altså i denne rapport IoT enheders som normalt betjenes manuelt, der nu er blevet tilkoblet internettet, dette kunne for eksemplet være kaffemaskiner, køleskabe eller mikrobølge ovne, der nu er koblet på nettet. Modsat PC'er og moderne smartphones er disse enheder typisk designet til enkelte formål, i stedet for at have brede anvendelsesmuligheder.

\newpage


