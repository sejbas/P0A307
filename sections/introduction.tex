\newpage
\chapter{Introduktion}
 
    Siden internettets oprindelse i 1960’erne, har dets indflydelsen i hverdagen kun vokset sig større og ud i fremtiden, ser det ud til kun at vokse mere eksplosivt, end førhen. Det er gået fra at være et simpelt eksperiment beregnet til at sende beskeder under kamp, til at have skabt millioner af jobs, og til at have forbundet over en tredjedel af verdensbefolkning.
    \\\\
Internettet blev opfundet i september 1969, af et hold forskere på UCLA (University of California, Los Angeles), efter de sammenkoblede en IMP (Interface message processor) til en SDS Sigma-7, hvilket var den første "node" (forbindelse) i det der dengang blev kaldet ARPA Net, som der senere blev til det, der i dag er kendt som internettet.\\
    Den første besked sendt fra ARPA Net (Advanced Research Projects Agency) var bogstavet I, den blev sendt fra UCLA til SRI (Science and Research Initiative) en måned efter etableringen af ARPA Net. 
    Den tredje besked var et g, hvilket fik hele systemet til at bryde sammen. Da man fik genoplivet systemet, var forbindelsen stabil, og efterfølgende inden året var omme opstod der et fungerende netværk bestående af 4 computere.
    Allerede i 1971 blev den første computervirus opfundet. Virussen blev kaldt creeper og den kopierede sig selv over netværket, hvorefter den gav beskeden ”I’m the creeper, catch me if you can”. Senere samme år blev den første e-mail introduceret af Ray Tomlinson.\\
    I oktober 1972 blev ARPANET for første gang demonstreret offentligt, under ICCC-konferencen, (International Computer Communication Conference) af Bob Metcalfe. I september 1973 blev ARPANET erstattet med Transmission Control Protocol/Internet Protocol (TCP/IP), hvorefter internettet blev globalt, og DARPA (Defense Advanced Research Projects Agency) åbnede 3 nye uafhængige TCP/IP-institutioner i Standford University, London University og BBN.
    1980, Tim Berners-Lee udvikler et program der registrerede links imellem computerne og projekterne i CERN. 
    Programmet blev kaldt ENQUIRE, og det var her ideen om World Wide Web begyndte at forme sig, og allerede efter 3 år viser en undersøgelse af Louis Harris and Associates Inc at 10\% af amerikanske voksne ejer en computer og at ud af dem har 14\% adgang til et modem som kan sende og modtage information, altså Ca. 1,4\% af voksne amerikanere havde adgang til internettet. 
    I marts 1985 åbnes den første kommercielle internet domæne af Symbolics.com, et computerfirma i Massachusetts, og
    i september 1988 åbnes det første Interhop trade show, hvilket er den første hjemmeside, der sammensætter 50 forskellige online forretninger på TCP/IP.\\
    Et par måneder senere introducerer Robert Tappan Morris, hvad der senere bliver kaldet The Morris Worm, hvilket var en computervirus der ramte og beskadigede Ca. 10\% af alle computere forbundet med internettet. Han var den første person som nogensinde blev dømt for online hærværk/svindel. Han startede senere et firma, som han solgte til Yahoo i 1998, han underviser i dag på MIT. I 1990 udvikles den første Internet Search Engine kaldet Archie af Alan Emtage, året efter udgiver Tim Berners-Lee koden for World Wide Web på internettet og den første hjemmeside på internettet kommer online fra SLAC National Accelerator Laboratory. November 1993, det første videokamera bliver tilsluttet til internettet i Cambridge University i London, hvilket er det første Webcam, og sommeren efter finder den første online transaktion sted, der bliver bestilt og betalt for en pizza fra Pizzahut.\\\\
    I 1995 fortæller en undersøgelse af The Pew Research Center at 14\% af alle voksne amerikanere er online, året efter udgiver Nokia, en mobiltelefon Nokia 9000 Communicator, den første telefon med en webbrowser. Senere sammen år introducerer Ethan Zuckerman den første pop-up reklame, hvilket kan senere undskyldte mange gange for, da dette blev kendt som ”The Internet’s Original Sin”. I år 1998 registrerer Google index’et 26 millioner webpages, 2 år senere bliver dette 1 milliard og 10 år senere 1 trillion. Marts 2007 Estonia bliver det første land, der bruger internettet til deres parlaments afstemning. December 2012, der bliver handlet for mere en 1 trillion dollars på internettet årligt, og mindre en to år senere er der mere en 3 milliarder mennesker verden over er på internettet.\autocite{gilpress2015}\\
    I løbet af internettets historie har vi fået tilkoblet meget andet en blot computere. Mobiltelefoner, TV og biler er også blevet koblet på internettet. Sidste nye er de såkaldte IoT (Internet of Things). Ideen om IoT er at alle fysiske ting i verden kan få indbygget en processor, og blive koblet på internettet, hvorfra enheden kan styres. Dette er ikke nødvendigvis kraftigt hardware, men kun lige nok til at forbinde netværket, og enheden. Der er ingen tvivl om at IoT enheder kan lave vores hverdag nemmere, og effektiviserer brugen af både tid og ressourcer og tid, men hvor længe vil hypen forsætte når nogle IoT producenter bliver ved med at ignorere de sikkerhedsmæssige problemer som er en del af deres produkter? Er det først efter brugeren er blevet ramt at brugeren begynder at overveje hvilke producenter de vælger at købe deres produkter fra? Et eksempel på nogen der først valgte at forlade IoT muligheden efter deres IoT enheder blev brugt imod dem, kunne være et østrigsk hotel, som blev offer for et \gls{phishing} angreb. Her kom hackerne ind i systemet ved hjælp af et falsk link som gave hackere adgang til hotellets LAN. herfra låste hackerne hotellets IoT døre. Herefter forlangte de 2 bitcoins for at give dem kontrollen tilbage. Dette kan dog ikke kun ske for firmaer. En dag vil en hacker måske få adgang til din smart-kaffekande og kræve 30kr for at du kan få din morgenkaffe. Derfor er det vigtigt at der bliver sat krav for sikkerheden på vores enheder som tilkobles på vores LAN.\\
   
    
    %Men for mange annoncerer hackerne ikke deres ankomst, her kommer de ind med formålet bare at indsamle data. Dette var noget man så hos et amerikansk kasino som havde et akvarie som selv stod for madning, \glspl{salinitet} og temperatur hvilket alt sammen blev styret over LAN. Hackerne kom så ind på den måde og brugte den adgang til at stjæle 10GB værd af data. I et event som dette kan hackerne sidde på dit netværk i lang tid før du får noget at vide, hvis det altså overhovedet opdages.\autocite{Examples}


\newpage