
\chapter{Rapport}

\section{Valg af emne}
Vi har valgt at arbejde med internet of things, da dette emne bliver stadig mere relevant i moderne hjem. Hjemmene fyldes af flere og flere enheder der alle skal være koblet på internettet på den ene eller anden måde.\\
Over de sidste ti år er der blevet tilføjet en lang række enheder til danskernes netværk. Det er ikke længere kun computere og printere der er tilkoblet nettet, nu rækker det sig også til telefoner, smartwatches, chromecasts, fjernsyn og så videre. \\
Listen over de ting som bliver tilkoblet netværk omkring i verden er stadig stigende og denne trend vil kun fortsætte fremover. Eftersom den vestlige verden bliver mere og mere afhængig af elektronik og dets "Always online" fordele, vil der også blive skubbet flere og flere af sådanne produkter på markedet, for at tjene på den stadig stigende efterspørgsel efter "smarte" produkter. Denne liste af produkter strækker sig efterhånden fra køleskabe til peberkværn, og så videre til toiletter. Alt bliver opkoblet til internettet og kan styres ved et tryk på en app. Alt dette lyder som en utopi, en verden hvori alt kan styres ved et tryk på en telefon, hvor man ikke længere skal rejse sig for at sætte kaffen over eller slukke for komfuret. En verden hvor alle husets goder, kan styres fra telefonen.\\
Problemet ligger dog lige netop i alle de enheder vi tilkobler vores netværk. Mange af disse enheder er lavet for at tjene penge på det stigende marked, og for at komme ud på markedet før konkurenterne, dette betyder ofte også at sikkerheden på disse enheder bliver nedprioriteret for at få et produkt ud på markedet hurtigere end konkurenterne.\\
Herved bliver der tilføjet en stor mængde af usikre enheder til ens netværk, hvilket giver et større potentiale for at ens lokale netværk bliver sårbart overfor uvedkommende enheder. Ovenpå at der bliver forbundet stadig flere enheder til ens netværk, er der i følge Forbes en stigning på 280\% på angreb rettet mod IoT enheder.\cite{website:Forbes2018} \\
Dette vil altså sige at disse enheder der bliver tilkoblet folks netværk, udgør en stadigt stigende trussel mod individernes egne netværk. Denne stigning af enheder gør sig ikke kun gældende i privaten, men bygger også ud imod en bredere undnyttelse af disse enheder i erhvervssektore af forskellige slags. \\

Kilde på mængde af IoT devices: \url{ttps://www.statista.com/statistics/471264/iot-number-of-connected-devices-worldwide/}


\subsection{Indskrænkning af IoT}
Da definitionen på Internet of Things er ret bred, har vi valgt at indskrænke vores definition af IoT enheder til følgende:


\subsection{Brainstorm}
Herunder er listet en række af problmer som vi kunne opsætte ved internet of things:
\begin{enumerate}
    \item Masser af hullede produkter på netværket.
    \item Et hul til at bryde det hele ned. (Routeren)
    \item Nemmere botnet.
    \item Remote control.
    \item Gammel software.
    \item Overvågning.
    \item Datamining.
    \item Bliver mere kompliceret.
    \item Afhænging af IOT.
    \item Yderlige afhænging af mobil.
    \item Besparelse på varme og strøm.
    \item Jobovertagelse. 
    \item Flere faktore kan give fejl på produktet.
\end{enumerate}

\subsection{Probleme analyse}

\subsubsection{Huller i netværket}

\subsubsection{Bliver mere kompliceret}

\subsubsection{Gammesl software}



\subsection{Problemstilling}