\section{Problemstilling}


Mange nye IoT enheder har, som beskrevet i de foregående sider, et problem med sikkerheden på enheden. Enhederne kan nogle gange være så usikre at de ikke er egnet til at være tilkoblet internettet, og er slet ikke udviklet med sikkerheden i fokus. Dog vil man som forbruger gerne have at sit produkt virker, og at sikkerhedsløsningen ikke er at fjerne noget af den funktionalitet som produktet er købt efter at skulle have. Derfor har rapporten fokus på at finde en løsning der kan afhjælpe sikkerheden, men ikke påvirker ydelsen af produktet. Det vil altså ikke bare være en mulighed at lukke for IoT enhedens eksterne kommunikation, da dette løser problemet, men stopper for at produktet fungere efter hensigten.\\
Problemet for brugeren er ikke nødvendigvis at kundens IoT enhed bliver hacket, men nærmere at adgangen til lige netop denne enhed, også åbner op for nemmere veje mellem de resterende enheder på det LAN hvortil den hackede IoT enhed er tilkoblet.

\section{Problemformulering}

Hvordan kan man sikre at eventuelle komprimerede IoT enheder ikke komprimere et helt netværk,uden at miste funktionalitet på IoT enhederne.\\
Er det muligt at tilføje en firewall der kan spore og stoppe komprimerede IoT enheder fra at komprimere resten af det LAN enheden er på.\\
Hvilke muligheder er der for at lukke forbindelsen mellem hackede enheder og personlige computere, uden at miste funktionalitet.\\

